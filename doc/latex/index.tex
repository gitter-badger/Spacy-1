\hypertarget{index_sec_intro}{}\section{Introduction}\label{index_sec_intro}
Spacy is a library that aims at
\begin{DoxyItemize}
\item providing an abstraction layer for the implementation of algorithms in a vector space setting (function space oriented algorithms)
\item providing a simple interface for sharing and comparing algorithms
\item having fun while developing algorithms
\item leaving you all freedom you want to implement your fancy algorithm (as long as you don\textquotesingle{}t violate the mathematical structure of vector spaces)
\end{DoxyItemize}\hypertarget{index_sec_overview}{}\section{Overview}\label{index_sec_overview}
The basis of Spacy is the abstraction layer for vector space settings. The interface of Spacy essentially builds on type erasure techniques provided by \href{http://www.boost.org/doc/libs/1_59_0/doc/html/boost_typeerasure.html}{\tt Boost.\+Type\+Erasure}. Moreover, C++11/14-\/features such as \href{http://en.cppreference.com/w/cpp/language/value_category}{\tt R-\/values} are used, so make sure to use the compiler option -\/std=c++1y.\hypertarget{index_sub_concepts}{}\subsection{Main concepts}\label{index_sub_concepts}

\begin{DoxyItemize}
\item First there is a vector space class (\hyperlink{classSpacy_1_1VectorSpace}{Spacy\+::\+Vector\+Space}), which models a \hyperlink{group__SpacyGroup_gabfc17f845736365ee79be8ce68d50b27_gabfc17f845736365ee79be8ce68d50b27}{Banach space}  (X,$\vert$$\vert$.$\vert$$\vert$) or a \hyperlink{group__SpacyGroup_ga63c49d211bf214be1fb321440ed03aad_ga63c49d211bf214be1fb321440ed03aad}{Hilbert space}  (X,(.,.)). Thus, vector spaces provide access to a \hyperlink{group__SpacyGroup_gaf4f33b11d657c48566d961a013c92bd1_NormAnchor}{norm} and possibly a \hyperlink{group__SpacyGroup_ga9fe0b4de20da1ab1ca3d04a0f96343e1_ScalarProductAnchor}{scalar product}. Moreover, vector spaces can be related to each other as primal or dual spaces. Eventually they can generate vectors.
\item Vectors (see \hyperlink{group__SpacyGroup_gafc144d2730ef87a67e54f8cd750b1f54_VectorAnchor}{Vector}, \hyperlink{group__ConceptGroup_gad6958389d1fa2758a8a64a0a24c36004_VectorConceptAnchor}{Vector\+Concept}) can be any class satisfying the \hyperlink{group__ConceptGroup_gad6958389d1fa2758a8a64a0a24c36004_VectorConceptAnchor}{Vector\+Concept}, i.\+e. vectors must be vectors in an almost pure mathematical sense.
\item Operator (see \hyperlink{group__SpacyGroup_ga3f89622eba80cf840b2a7102f1303455_OperatorAnchor}{Operator}, \hyperlink{group__ConceptGroup_ga7d984281b647a6d8e4c0a7ea5faeb90e_OperatorConceptAnchor}{Operator\+Concept}) are mappings between different vector spaces.
\item Functionals (see \hyperlink{group__SpacyGroup_ga673218f603c93790864aef12c89d3a35_FunctionalAnchor}{Functional}, \hyperlink{group__ConceptGroup_ga5273b50bd3e8c9a3f5a1e6b5f170836d_FunctionalConceptAnchor}{Functional\+Concept}) are mappings from a vector spaces into the space of real numbers $ \mathbb{R} $.
\end{DoxyItemize}\hypertarget{index_sec_usage_fenics}{}\section{Usage with F\+Enics}\label{index_sec_usage_fenics}
Examples for nonlinear P\+D\+Es and optimal control problems with \href{http://www.fenicsproject.org}{\tt F\+Eni\+C\+S} are given in Examples/\+F\+Eni\+C\+S.\hypertarget{index_sub_usage_fenics_pde}{}\subsection{Nonlinear P\+D\+Es with F\+Eni\+C\+S}\label{index_sub_usage_fenics_pde}
For an operator equation $A(x)=0$, discretized with F\+Eni\+C\+S (\textquotesingle{}L\textquotesingle{} denoting the residual form and \textquotesingle{}a\textquotesingle{} the gradient form), simplest usage is as follows 
\begin{DoxyCode}
...
MyFEniCSExample::FunctionSpace V\{mesh\}
MyFEniCSExample::LinearForm L\{V\};
MyFEniCSExample::BilinearForm a\{V,V\};
...

\textcolor{comment}{// Create function space.}
auto domain = \hyperlink{group__FenicsGroup_ga89defe8c7e08ab224af2a3cd0445e254_ga89defe8c7e08ab224af2a3cd0445e254}{Spacy::FEniCS::makeHilbertSpace}(V);

\textcolor{comment}{// Create operator mapping into the dual space of domain (which is, due to the Hilbert space structure,
       associated with domain itself).}
\textcolor{comment}{// You can also specify the range space if it differs from the dual space of domain}
\textcolor{comment}{// This is illustrated in the PDE example for Kaskade 7}
\textcolor{keyword}{auto} A = \hyperlink{group__KaskadeGroup_ga0e8d7d2c51e429e22561ef813fc97589_ga0e8d7d2c51e429e22561ef813fc97589}{Spacy::FEniCS::makeOperator}( L , a , domain );

\textcolor{comment}{// Solve with covariant Newton method with initial guess x0=0.}
\textcolor{keyword}{auto} x = \hyperlink{group__NewtonGroup_gab9d1c5b64e93d7ac051f8a7b41bf520a_gab9d1c5b64e93d7ac051f8a7b41bf520a}{Spacy::covariantNewton}(A);


\textcolor{comment}{// Copy solution back to dolfin::Function.}
dolfin::Function u(V);
\hyperlink{group__FenicsGroup_ga7f43f0c660d0646adb031b453c536bb0_ga7f43f0c660d0646adb031b453c536bb0}{Spacy::FEniCS::copy}(x,u);
...
\end{DoxyCode}
\hypertarget{index_sec_usage_kaskade}{}\section{Usage with Kaskade 7}\label{index_sec_usage_kaskade}
Examples for nonlinear P\+D\+Es and optimal control problems with \href{http://www.zib.de/projects/kaskade7-finite-element-toolbox}{\tt Kaskade 7} are given in Examples/\+Kaskade.\hypertarget{index_sub_usage_kaskade_pde}{}\subsection{Nonlinear P\+D\+Es with Kaskade 7}\label{index_sub_usage_kaskade_pde}
For an operator equation $A(x)=0$, discretized with Kaskade 7 and with another scalar product, usage is as follows 
\begin{DoxyCode}
...
auto space = Kaskade::FEFunctionSpace< ContinuousLagrangeMapper<double,LeafView> >\{(\} gridManager , 
      gridManager.grid().leafView() , order \};
...
auto variableSetDescription = VariableSetDescription\{ spaces , \{\textcolor{stringliteral}{"x"}\} \};
\hyperlink{group__SpacyGroup_ga673218f603c93790864aef12c89d3a35_ga673218f603c93790864aef12c89d3a35}{Functional} F\{ ... \};
...

\textcolor{comment}{// Create domain and range space.}
auto domain = \hyperlink{group__KaskadeGroup_ga04d45446864bbf87770d02eade7b64cf_ga04d45446864bbf87770d02eade7b64cf}{Spacy::Kaskade::makeHilbertSpace}( space );
\textcolor{keyword}{auto} range  = \hyperlink{group__KaskadeGroup_ga04d45446864bbf87770d02eade7b64cf_ga04d45446864bbf87770d02eade7b64cf}{Spacy::Kaskade::makeHilbertSpace}( space );

\textcolor{comment}{// Create operator}
\textcolor{keyword}{auto} A = \hyperlink{group__KaskadeGroup_ga0e8d7d2c51e429e22561ef813fc97589_ga0e8d7d2c51e429e22561ef813fc97589}{Spacy::Kaskade::makeOperator}( F , domain , range );

\textcolor{comment}{// Set induced scalar product on domain space.}
\textcolor{keyword}{auto} x0 = domain.vector();
domain.setScalarProduct( InducedScalarProduct( A.linearization(x0) );

\textcolor{comment}{// Solve with covariant Newton method with initial guess x0=0.}
\textcolor{keyword}{auto} x = \hyperlink{group__NewtonGroup_gab9d1c5b64e93d7ac051f8a7b41bf520a_gab9d1c5b64e93d7ac051f8a7b41bf520a}{Spacy::covariantNewton}( A , x0 );


\textcolor{comment}{// copy solution back to dolfin::Function}
\textcolor{keyword}{typename} VariableSetDescription::VariableSet u( variableSetDescription );
\hyperlink{group__KaskadeGroup_gafbbd7e385eda54f651c45b1c074d7bf9_gafbbd7e385eda54f651c45b1c074d7bf9}{Spacy::Kaskade::copy}( x , u );
...
\end{DoxyCode}
 